\section{Метод решения}
Алгоритм реверса вектора, реализованный в лабораторной работе, осуществляет разворот переданного вектора inplace, результат функции записывается в тот же вектор. Для этого каждый элемент вектора необходимо поменять местами с симметричным относительно середины вектора. Индекс симметричного элемента вычисляется по формуле: \textit{длина вектора - индекс элемента - 1}. Очевидно, что такие замены нужны для всех элементов вектора вплоть до его середины. Сложность алгоритма по времени --- $O(n)$, по памяти --- $O(n)$. Каждый поток обрабатывает элемент вектора со своим индексом, затем все элементы с шагом, равным суммарному числу потоков.

\section{Описание программы}
В файле \texttt{lab1.cu} расположен основной код программы. Функция \texttt{reverse} --- вычислительное ядро --- осуществляет описанный выше алгоритм и принимает два аргумента: указатель на массив данных и размер массива. Файл \texttt{error\_checkers.hpp} содержит вспомогательные функции и макросы для обработки ошибок CUDA.
