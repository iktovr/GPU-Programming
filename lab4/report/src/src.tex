\section{Метод решения}
Для решения квадратной СЛАУ необходимо привести матрицу системы к верхнетреугольному виду, при помощи прямого хода метода Гаусса, затем найти решение обратным ходом ($x_n$ тривиально находится из единственного элемента последней строки, $x_{n-1}$ --- через $x_n$ и так далее).

Один шаг метода Гаусса состоит из нескольких этапов: выделение ведущего элемента (максимального в столбце), перестановка строки с ведущим элементом на первое место для данного шага, зануление текущего столбца. Суммарно необходимо выполнить $n$ шагов.

Сложность прямого хода --- $O(n^3)$, обратного --- $O(n^2)$. Итого --- $O(n^3)$.

\section{Описание программы}
В файле \texttt{lab4.cu} расположен основной код программы. Функция \texttt{gaussian\_solver\_step} --- вычислительное ядро --- осуществляет финальный этап шага метода Гаусса. Функция \texttt{swap\_rows} --- другое ядро --- переставляет местами две строки матрицы. Максимальный элемент столбца находится при помощи функции \texttt{thrust::max\_element}. Таким образом весь прямой ход метода Гаусса осуществляется над матрицей в видеопамяти. Обратный ход не распараллеливается, так как его сложность значительно меньше, и целиком выполняется на CPU.
