\section{Выводы}
В ходе выполнения лабораторной работы я познакомился с параллельной реализацией метода Гаусса. При работе с матрицами сильно ощущается эффект распараллеливания, так как очень большое количество матричных операций имеют кубическую сложность.

Я наглядно убедился в важности объединения запросов к глобальной памяти --- если его не происходит, время выполнения алгоритма увеличивается в 2-3 раза. Оптимальная конфигурация ядра: сетка блоков --- 32$\times$32, сетка потоков --- 32$\times$32.
\pagebreak
