\pagebreak
\section{Результаты}

\textbf{Анализ программы профилировщиком}

Конфигурация ядра: 32$\times$32, 32$\times$32; размер тестовой матрицы: 2000$\times$2000. Из вывода \texttt{nvprof} убраны нечитаемые названия ядер \texttt{thrust}.

В ядре \texttt{gaussian\_solver\_step} потоки варпа обрабатывают столбцы матрицы:
\begin{lstlisting}[frame=none, numbers=none, keepspaces=true, language=]
Invocations                                Event Name      Min      Max      Avg
Device "GeForce GT 545 (0)"
        Kernel: gaussian_solver_step(double*, int, int)
       2000                  global_store_transaction        0   500208   161671
        Kernel: swap_rows(double*, int, int, int)
       1994                  global_store_transaction     6144     6144     6144
\end{lstlisting}

В ядре \texttt{gaussian\_solver\_step} потоки варпа обрабатывают строки матрицы:
\begin{lstlisting}[frame=none, numbers=none, keepspaces=true, language=]
Invocations                                Event Name      Min      Max      Avg
Device "GeForce GT 545 (0)"
        Kernel: gaussian_solver_step(double*, int, int)
       2000                  global_store_transaction        0  4002432  1335062
        Kernel: swap_rows(double*, int, int, int)
       1994                  global_store_transaction     6144     6144     6144
\end{lstlisting}

Видно, что во втором случае происходит гораздо больше обращений к глобальной памяти, т.е. объединения запросов не происходит. На практике это приводит к увеличению времени работы в 2-3 раза. 

\textbf{Сравнение времени работы}
\begin{center}
\begin{tabular}{|l*{6}{|r}|}
\hline
\textbf{Размер матрицы} & 100 & 500 & 1000 & 2000 & 5000 & 10000 \\
\hline
\hline
\textbf{Конфигурация} & \multicolumn{6}{c|}{\textbf{Время выполнения, мс}} \\
\hline
CPU                          &  14.8886 &  788.532 & 5958.03 & 47452.10 & 745217 &      - \\
\hline
1$\times$1, 32$\times$32     &  36.4768 &  256.299 & 1145.10 &  8840.49 & 114357 &      - \\
\hline
32$\times$32, 32$\times$32   &  53.7845 &  287.704 &  932.85 &  4461.41 &  43472 & 411956 \\
\hline
64$\times$64, 32$\times$32   &  99.7329 &  547.723 & 1513.98 &  6305.97 &  51747 & 466804 \\
\hline
128$\times$128, 32$\times$32 & 284.3100 & 1559.010 & 3640.97 & 10970.80 &  73901 & 518470 \\
\hline
\end{tabular}
\end{center}
