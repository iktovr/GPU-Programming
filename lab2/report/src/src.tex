\section{Метод решения}
Для выделения границ изображения необходимо осуществить на каждом его пикселе две операции свертки, по одной на компоненту вектора градиента. При этом свертка осуществляется не по исходным значениям цвета, а по величине яркости, вычисляемой по формуле $u = 0.299r +  0.587g + 0.114b$. По полученным компонентам находится модуль градиента, его значение необходимо ограничить максимальной величиной компоненты цвета -- 255. В пиксель выходного изображения записывается модуль градиента во все цветовые каналы, альфа канал не изменяется. Сложность алгоритма --- $O(w \cdot h)$.

\section{Описание программы}
В файле \texttt{lab2.cu} расположен основной код программы. Функция \texttt{sobel\_filter} --- вычислительное ядро --- осуществляет описанный выше алгоритм и принимает четыре аргумента: текстурный объект, указатель на массив данных выходного изображения и размеры изображения. Ядра свертки записаны в константную память, так как к ним осуществляется много обращений.