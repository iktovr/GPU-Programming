\section{Выводы}
В ходе выполнения лабораторной работы я познакомился с обработкой изображений при помощи механизма свертки. Сверточные фильтры используются как непосредственно для обработки, так и, например, в компьютерном зрении.

В процессе выполнения работы я столкнулся с тем, что текстурные объекты доступны только на GPU с compute capability $\geqslant$ 3.0, в связи с чем мне пришлось тестировать программу на другом устройстве. Результаты тестирования в целом не изменились: время выполнения падает при увеличении числа потоков, но затем снова начинает возрастать. Единственное отличие: даже использование только одного варпа значительно сокращает время работы, по сравнению с исполнением на CPU. Оптимальная конфигурация ядра: сетка блоков --- 32x32, сетка потоков --- 32x8.
\pagebreak
