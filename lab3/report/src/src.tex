\section{Метод решения}
Для метода максимального правдоподобия необходимы три величины для каждого класса $i$: выборочное среднее $avg_i$, обратная матрица и определитель матрицы ковариации $conv_i$. Все эти величины предподсчитываются заранее, для вычисления определителя и обратной матрицы используются явно заданные формулы, так как матрицы имеют размер 3x3. Для определения класса пикселя $p$ используется дискриминантная функция следующего вида:
$$-(p - avg_i)^{T} \cdot conv_i^{-1} \cdot (p - avg_i) - \ln(|\det conv_i|)$$
пиксель принадлежит тому классу, при котором дискриминантная функция максимальна. Сложность алгоритма --- $O(n \cdot w \cdot h)$, где $n$ -- количество классов.

\section{Описание программы}
В файле \texttt{lab3.cu} расположен основной код программы. Функция \texttt{maximum\_likelihood} --- вычислительное ядро --- осуществляет описанный выше алгоритм и принимает три аргумента: указатель на массив данных и размер изображения и количество классов. Нужные алгоритму данные о классах --- выборочное среднее и обратная матрица ковариации и определитель матрицы ковариации --- записаны в константную память. Файл \texttt{matrix.hpp} содержит классы векторов (для CPU реализации) и матрицы с необходимыми функциями и операторами.
