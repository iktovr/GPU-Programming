\section{Выводы}
В ходе выполнения лабораторной работы я познакомился с классификацией пикселей изображения. Алгоритмы из данной лабораторной могут использоваться как непосредственно для классификации, так и для сжатия изображения путем ограничения палитры. Однако, использование не более 32 оттенков приводит к не очень хорошим результатам.

Метод максимального правдоподобия (как и другие методы, использующие матрицу ковариации) отличается тем, что ему необходимо минимум четыре различных (не лежащих на одной плоскости в пространстве цветов) пикселя для каждого класса --- нельзя задать конкретные цвета и классифицировать только по ним. Это усложняет тестирование на реальных изображениях. Один из выходов --- ручное выделение областей, из которых составляются выборки классов.

Замеры времени в этот раз отличаются тем, что, начиная с конфигурации 128,128, при дальнейшем увеличении числа потоков время изменяется очень медленно. 128,128 является оптимальной конфигурацией.

\pagebreak
