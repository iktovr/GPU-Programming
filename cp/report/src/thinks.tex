\section{Выводы}
В ходе выполнения курсовой работы я познакомился с алгоритмом обратной трассировки лучей и его параллельной реализацией. Трассировка лучей позволяет получать фотореалистичные изображения, но крайне затратна вычислительно. Число лучей, которые необходимо обрабатывать потенциально экспоненциально растет, однако на практике остается порядка числа пикселей изображения, что все равно довольно много при большом разрешении или использовании сглаживания.

Одной из самых сложных частей работы было добиться того, чтобы максимальный объем программы не зависел от версии трассировки и параллельно постоянно добавлять все новый и новый функционал для соответствия требованиям.

Результаты тестирования вполне ожидаемы. Даже небольшое распараллеливание хорошо ускоряет трассировку. 
\pagebreak
