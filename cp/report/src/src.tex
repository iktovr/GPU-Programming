\section{Метод решения}
В курсовой работе реализован алгоритм обратной трассировки лучей с фиксированным числом переотражений и без явного использования рекурсии. Основная единица исполнения трассировки лучей --- трассировка одного луча (вычисление точки его пересечения со сценой), вычисления цвета в данной точке и создание отраженных или преломленных лучей. Исходные и новые лучи можно хранить в двух массивах и таким образом, одна итерация трассировки всех лучей соответствует одному рекурсивному спуску.

Для вычисления цвета в точке пересечения используется затенение по Фонгу, согласно которому он складывается из цветовых интенсивностей трех компонент освещения: фоновой, рассеянной и бликовой
$$I = I_a + I_d + I_s$$

Фоновая компонента:
$$I_a = K_a i_a$$
где $K_a$ --- способность материала воспринимать фоновое освещение, $i_a$ --- интенсивность фонового освещения.

Рассеянная компонента:
$$I_d = K_d \cos(\vec{L}, \vec{N}) i_d = K_d (\vec{L} \cdot \vec{N}) i_d$$
где $K_d$ --- способность материала воспринимать рассеянное освещение, $i_a$ --- интенсивность рассеянного освещения, $\vec{L}$ --- направление из точки на источник света, $\vec{N}$ --- нормаль в точке.

Бликовая (зеркальная) компонента:
$$I_s = K_s \cos^p(\vec{R}, \vec{V}) i_s = K_s (\vec{R} \cdot \vec{V})^p i_s$$
где $K_s$ --- способность материала воспринимать бликовое освещение, $p$ --- коэффициент блеска, $i_a$ --- интенсивность бликового освещения, $\vec{R}$ --- направление отраженного луча, $\vec{V}$ --- направление из точки на наблюдателя.

Для создания теней интенсивность источника света в точке корректируется с учетом коэффицента пропускания и цвета материалов, через которые проходит луч на пути к этой точке.

В качестве алгоритма сглаживания используется SSAA --- изображение рендерится в $N$ раз большем размере, итоговое изображение получается усреднением цвета в непересекающихся окнах $N \times N$.

\section{Описание программы}
\subsubsection*{Сцена}
Все объекты сцены хранятся в различных массивах. А именно сцена содержит: массив материалов, массив вершин, массив треугольников, массив сфер, массив данных об полигональных объектах, массив источников света, массив данных о текстурах и, наконец, сами текстуры так же в одном массиве.

Полигональный объект содержит индексы треугольников, сфер и ограничивающую сферу. Каждый треугольник хранит индексы вершин, индекс материала и текстуры, а также текстурные координаты каждой вершины. Так как для всей объектов одного типа используется сквозная индексация, конкретные индексы (треугольников, вершин, материалов и т.д.) обновляются при добавлении объектов в сцену.

Структура сцены также содержит методы для нахождения пересечения луча, расчета интенсивности света в какой-либо точке и получения пикселя текстуры.

Массивы хранятся в виде указателей, что позволяет одинаково работать со сценой содержащейся как в RAM, так и в видеопамяти.

Необходимые многогранники загружаются из файлов \texttt{obj}. Внутренние \enquote{источники света} расставляются параллельно линиям, специально оставленных в моделях, и сохраняются в массиве сфер.

\subsubsection*{Трассировка лучей}
Трассировка лучей на GPU реализована без использования рекурсии. Управляющий функция \texttt{render} вызывает инициализацию лучей, затем, в цикле, трассировку текущего массива лучей. После каждой итерации массивы лучей меняются местами, и если лучей стало слишком много, происходит перевыделения памяти для результирующего массива.

Инициализация и трассировка выполняются параллельно. При трассировке дочерние лучи записываются на первый свободный индекс массива, который обновляется атомарным сложением; после получения цвета в точке пересечения, он так же добавляется к пикселю изображения атомарно.

Версия для CPU реализована с использованием рекурсии. В функции \texttt{render} для каждого пикселя создается луч, трассировку которого производит функция \texttt{ray\_color}, осуществляя при необходимости рекурсивные вызовы для отраженного или преломленного луча.

Функции для определения пересечения реализованы как методы сцены и примитивов. При проверке пересечения с объектом используется небольшая оптимизация --- проверка на пересечение с ограничивающей его сферой. Параметры сферы определяются при добавлении объекта в сцену.

\subsubsection*{Освещение}
Основной цвет в точке пересечения вычисляется с использованием затенения по Фонгу, описанного ранее. Для учета теней, интенсивность света вычисляется в каждой точке при помощи трассировки луча от источника до этой точки и аккумулирования коэффициентов прозрачности и цветов материалов при каждом пересечении. Также, каждый луч содержит собственный коэффициент затухания, который обновляется при отражении или преломлении луча и учитывается для итогового цвета.

\subsubsection*{Текстурирование}
Для наложения текстур используются текстурные координаты. Углы прямоугольной текстуры имеют координаты $(1, 1)$, $(1, 0)$, $(0, 0)$, и $(0, 1)$ соответственно. Вершина каждого треугольника имеет свои текстурные координаты. Текстурные координаты точки внутри треугольника определяются при помощи интерполяции текстурных координат вершин с использованием барицентрических координат это точки. Такой способ позволяет текстурировать треугольники с произвольным положением в пространстве.

\pagebreak