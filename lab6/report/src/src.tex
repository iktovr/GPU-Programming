\section{Метод решения}
Для обработки коллизий между сферами, а также сфер с камерой, пулей и стенами используется электростатика --- каждая сфера считается имеющей некоторый заряд, камера и пуля --- больший заряд того же знака. Для препятствия прохождения сфер сквозь стены, каждая сфера дополнительно отталкивается от заряда такого же размера, спроецированного на каждую из стен.

\section{Описание программы}
Взаимодействие сфер рассчитывается самым примитивным образом --- каждой с каждой. Для отрисовки карты напряженности так же необходимо для каждого пикселя учесть вклад каждой сферы. Это две самых вычислительно сложных части обновления всех объектов, поэтому они распараллеливаются. Во взаимодействии сфер параллельно рассчитывается результирующая скорость сферы, в карте напряженности --- цвет пикселя.
