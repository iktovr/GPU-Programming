\section{Метод решения}
Карманная сортировка сортирует массив путем распределения его по карманам, и сортировки каждого кармана по отдельности. Разделение по карманам осуществляется в два этапа: сначала массив распределяется по маленьким карманам (распределение происходит из предположения, что данные распределены равномерно), затем маленькие карманы объединяются в большие, к которым непосредственно применяется другой алгоритм сортировки. Размер большого кармана выбирается таким образом, чтобы его можно было отсортировать в разделяемой памяти. Если карман получился больше нужного размера, к нему рекурсивно применяется алгоритм с самого начала.

Для распределения элементов по карманам необходимо знать минимум и максимум в массиве, которые можно найти алгоритмом редукции. Перегруппировка массива в соответствии с карманами есть ни что иное, как модифицированная сортировка подсчетом, поэтому ее так же можно выполнить при помощи алгоритма гистограммы и алгоритма scan для суммы. Сортировка больших карманов осуществляется битонической сортировкой.

Сложность работы карманной сортировки в среднем линейная. Наилучшие результаты достигаются если числа в массиве равномерно распределены.

\section{Описание программы}
Программа разбита на заголовочные файлы, соответствующие основным алгоритмам --- редукции, scan, алгоритму гистограммы, битонической сортировки и карманной сортировки. Файл \texttt{utils.hpp} содержит дополнительные математические и вспомогательные функции.

Реализация редукции и scan позволяет использовать их вместе с любой бинарной функцией, однако в C++11 это не очень удобно делать, из-за отсутствия шаблонов переменных. Битоническая сортировка реализована полностью, для работы с массивом любой длины, хотя используется только для сортировки в разделяемой памяти.

Редукция использует все основные оптимизации, кроме развертки цикла; scan реализован без оптимизаций, кроме использования фиктивных элементов; алгоритм гистограммы всегда строит ее в глобальной памяти; битоническая сортировка оптимизирована для небольших массивов и не дополняет их до степени двойки в глобальной памяти --- это значительно уменьшает время работы карманной сортировки.
